
\section{Non-Interacting Bose-Gas}
The theory of the non-interacting Bose Gas is developed in full in most texbooks on statistical mechanics, 
so the reader is referred to them for more information about how to calculate the proper statistics. We only want to 
recall the results briefly.

\subsection{Bose distribution}
By considering a grand canonical ensemble for bosons~\cite{huang}
one arrives at the familiar energy  distribution function:
\begin{equation}\label{eq:Bosedist}
	f(\varepsilon_{n})=\frac{1}{e^{\frac{\varepsilon_n-\mu}{k_BT}}-1}
\end{equation}
Where $\varepsilon_n$ denotes the Energies of the single-particle eigenstates of the system, and $\mu$ is the chemical 
potential. For high temperatures, 
the chemical potential is considerably smaller than the minimal energy of the system, 
as there are very few particles in the same state. As temperature
falls however, the chemical potential gets comparable to this minimal energy, yet cannot exceed it for the reason that 
equation~(\ref{eq:Bosedist}) would 
be negative, which is clearly non-physical. As a consequence, the occupation number of any excited single-particle 
state cannot exceed the value 
$\frac{1}{e^{\varepsilon_n-\varepsilon_{min}/k_BT}-1}$. This accounts for the fact that if we consider $N$ particles
of which less than $N$ are in an excited state, the remaining particles have to be accomodated in the lowest state, 
the system is then said to exhibit Bose-Einstein
condensation.

\subsection{The Bose-Einstein Condensate in a Harmonic Trap}
Bose-Einstein Condensates are usually produced inside a magnetic or optical trap. The first BEC was cooled using
laser-cooling for the atoms to be slow enough to be held in a magnetic (quadrupole) trap. For details and various
implementations of several traps, the reader is referred to~\cite{pethick}, as well as to the original article 
by Anderson et. al.~\cite{anderson}. 
The traps usually introduce a harmonic potential to the overall Hamiltonian of the system. In our case, we consider 
a potential of the form:
\begin{equation}\label{eq:trap}
	V(x)=\frac{x^2}{2}
\end{equation}
Which is the potential one would find for a normal harmonic oscillator. Indeed, if we consider single-particle
states of bosons and choose units so that $\hbar=1$ and the mass $m=1$, the Hamiltonian for a single particle (and hence for an 
arbitrary number of particles all
in the same single particle ground state) reads:
\begin{equation}\label{eq:trapham}
	\hat{H}=-\frac{1}{2}\frac{d^2}{dx^2}+V(x)
\end{equation}
The solution to the  Schr\"odinger equation for this System is familiar from standard Quantum Mechanics textbooks
\footnote{A very thorough introduction is given in~\cite{schwabl}}, 
and it suffices to state that the Groundstate
of this system is described in position space by the wave-function:
\begin{equation}\label{eq:HOsol}
	\frac{1}{\pi^{1/4}}e^{-\frac{x^2}{2}}
\end{equation}
which is already normalized to unity. The fact that this system is soluble analytically provides an excellent means of 
checking the results of the numerical simulations for correctness of the code. However, this treatment does not account for the fact that in real condensates,
there are actually interactions between the particles and that these interactions are very peculiar to the nature of the atoms in the condensate. As this system has
stationary states, they do not change and the condensate would therefore stay in exactly this same state. Moreover, as the minimal energy is given and all 
the particles would come to the minimal energy state, the system would be in its ground state, having no fluctuations at all. In this work, we consider a one-dimensional condensate. Such a system was only recently (2003)
realized~\cite{moritz}, which gives the system at hand a practical relevance. It is different from the three-dimensional condensate in that it cannot exist at
finite temperatures. This fact is a direct consequence of the Mermin-Wagner-Hohenberg Theorem, which basically states that there cannot be any spontaneous symmetry
breaking at finite temperature for one- or two-dimensional systems. A proof for the theorem can be found in the original
work~\cite{mermin}, applied to
magnetic systems. Two years later,
Hohenberg~\cite{hohenberg} pointed out that in the case of a superliquid, the theorem also applies since in a superliquid (and in a Bose-Einstein condensate, for
that matter), all particles carry exactly the same momentum, which means that the symmetry of such a system is effectively broken. This however cannot occur for a
one-dimensional system at finite temperature. For the calculations, I refer the reader to the above cited articles. 

\subsection{The Bose-Einstein Condensate in an Optical Lattice}\label{sec:lattice}
Instead of only confining the Bose-Einstein condensate in a trap of the form~(\ref{eq:trap}), it is also possible to use a potential of the form 
\begin{equation}\label{eq:trap}
	\hat{V}_{\textrm{\small lattice}} = s\frac{q_B^2}{2}\sin^2(qx)
\end{equation}
where $s$ is some parameter to control the intensity of the lattice. As the name suggests, the lattice is formed by means of lasers, the electric field will be of
the form $E_0\sin(qx)sin(\omega t)$, where $q$ is just the wave vector of the incident laser
light, $q=\frac{2\pi}{\lambda}$. As we will see, Bose-Einstein Condensates follow the Gross-Pitaevskii equation~(\ref{eq:GrossPitaevskii}). 
A solution for the
Ground-state for this equation in the one-dimensional case can be given in terms of its Fourier series:
\begin{equation}\label{eq:latticegs}
	\psi_0=\sum_{l=0,\pm 1, \ldots}\psi_le^{\frac{il2\pi x}{d}}
\end{equation}
$d$ being the lattice constant, $d=\frac{\lambda}{2}$, and $\psi_l=\frac{1}{d}\int_{-\frac{d}{2}}^{\frac{d}{2}}\psi_0e^{-\frac{il2\pi x}{d}}dz$, as usual. The behaviour of
the wave function is best shown in momentum space. Thus, we form the Fourier transform of the above function, again using units such that $\hbar=1$:
\begin{eqnarray}
	\psi_0(p)&=&\int_{-\infty}^{\infty}\sum_{l=0,\pm 1, \ldots}\psi_le^{\frac{i2\pi l x}{d}}e^{-ipx}dx\nonumber\\
		{}&=&\sum_{l=0,\pm 1, \ldots}\psi_l\int_{-\infty}^{\infty}e^{-ix(p-\frac{i2\pi l x}{d})}dx\nonumber\\
		{}&=&2\pi\sum_{l=0,\pm 1, \ldots}\psi_l\delta(p-\frac{2\pi l}{d})\nonumber
\end{eqnarray}
Thus, it is seen that the momentum distribution is given by a sum of delta peaks. However, we want to consider a condensate that is kicked, so we have to incorporate some means of kicking, which is
experimentally done by tilting the trap out of the axis for a short time. Therefore, apart from the lattice, we also have to add a trap. This however is just the superposition of 
two potential terms. 
The lattice in the potential was added by simply adding the term $\frac{q_B^2}{2}\sin^2(q_Bx)$~\cite{menotti}, where $q_B$ is the wavevector 
at the boundary of the first Brillouin zone, $q_B=\frac{\pi}{a}$ and $a$ is the lattice constant. The Groundstate of such a potential is shown 
in the figures below. 
The form of the lattice is not arbitrary. The factor $\frac{q_B^2}{2}$ arises, because of the recoil energy an atom acquires when absorbing a 
photon with momentum $q$. As the field produced by a laser which is normally used to produce a lattice is $E_0\sin(qx)\sin(\omega t)$, the time-averaged 
field is then proportional to $\sin^2(qx)$~\cite{pitaevskii}.
It has to be noted, that the dynamics are non-trivial, because of the nonlinearity that arises from the interatomic interactions. It is most fitting to 
use numerical techniques. The following pictures will illustrate the nature of the Ground state further. 
Let now the potential be of the form above, and let the atoms be trapped with the normal harmonic potential. The overall external potential is then periodic. 
Let again $\psi(x)$ be the wavefunction of a one dimensional Bose-Einstein condensate still obeying the equation~(\ref{eq:GrossPitaevskii}). As is explained above, Figures~\ref{fig:WFk0g1} and~\ref{fig:WFk0g5} show the first two sattelite peaks, that 
are obtained in precisely this way.
\begin{figure}[H]
\begin{center}
\includegraphics[scale=0.5]{figures/WFk0g1}
\caption{Ground state distribution of a Bose-Einstein condensate in a periodic potential and its corresponding momentum distribution with
		$g=1$}
\label{fig:WFk0g1}
\end{center}
\end{figure}


\begin{figure}[H]
\begin{center}
\includegraphics[scale=0.5]{figures/WFk0g5}
\caption{Same as figure \ref{fig:WFk0g1} with $g=5$}
\label{fig:WFk0g5}
\end{center}
\end{figure}



		
