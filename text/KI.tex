\section{Kicked Condensates}
A kick in theory is introduced by a delta peak~\cite{reichl}, having a given kick strength $K$.
In our split-step operator framework, this is equal to multiplying the wavefunction with a phase factor, 
thereby leaving its norm untouched.
My starting point was the delta-kicked rotor, already analyzed in~\cite{reichl} and recently treated quantum mechanically in~\cite{zhang}, where
it was shown using a split-step operator technique, that for larger couplings between the atoms, the system undergoes a
transition to chaos when kicked periodically. Such a system is very close to the above mentioned delta-kicked rotor in that
as the interactions in the mean-field approximation add a nonlinear term to the Schr\"odinger equation. This will render the
system open to chaotic, or instable behaviour. Chaos is usually not possible in Quantum Mechanics, because of the linearity
of the Schr\"odinger equation, and it is generally disputed whether or not one can speak of chaos in Quantum Mechanics,
since the classical definition, e.g. via Lyapounov-exponents~\cite{arnold}, requires some notion of a trajectory, which is
naturally absent in Quantum Mechanics. However, a nonlinear Schr\"odinger equation exhibits exponential sensitivity to
initial conditions. As~\cite{zhang} shows, a one-dimensional Bose-Einstein Condensate which is confined to a ring and is then
periodically kicked, can exhibit quantum anti-resonance (periodic recurrence between different states), which is destroyed
for large enough coupling constants that render the nonlinearity appreciable. Figure~\ref{fig:zhang} shows quantum anti-resonance and its disappearance when the coupling becomes strong enough

\begin{figure}[H]
\begin{center}
\includegraphics[scale=0.6]{figures/zhang}
\caption{Average energy of a BEC confined to a ring. The nonlinearity destroys quantum anti-resonance. Taken from~\cite{zhang}}
\label{fig:zhang}
\end{center}
\end{figure}

This work examins the effects of a condensate in a trap with an optical lattice superimposed. The condensate is kicked 
once and subsequently left to its own and it is assumed that the Gross-Pitaevskii equation still holds. The propagation is done using a split-step operator technique both for finding the ground state and
propagating the system in time. The result will be phase decoherence in momentum-space, owing to the nonlinear term
in the Gross-Pitaevskii equation, if only the kick that is exerted is strong enough for a given coupling constant.
For our purposes, tne kick was given by $K\sin(x)\delta(t_k)$, which could be realized in a 
physical system by tilting the trap in which the
condensate is contained out of place for a short time. When converting this into an exponential operator, it is seen that 
for the split-step technique applied to solve this system, we have to multiply
our wave function by the factor $e^{-iK\sin(x)}$, since the kick is taken to be instantaneous at an arbitrary chosen time $t_k$. 
As can be analytically shown in the case of the Gross-Pitaevskii equation 
with no interaction between the atoms, this shifts the momentum-distribution to the right:
Our analytical system reads:
\begin{equation}\label{eq:HO}	
	\left(-\frac{1}{2}\frac{d^2 \psi}{dx^2}+\frac{x^2}{2}\right)\psi = E\psi
\end{equation}
This has the very familiar solution $\psi=\frac{1}{\pi^{\frac{1}{4} } }e^{-\frac{x^2}{2} }$. The mean momentum after applying a kick is obtained by observing that:
\begin{eqnarray}
\langle p \rangle & = & \frac{1}{i\pi^{\frac{1}{2} } }\int_{-\infty}^{\infty}(x+iK\cos x)e^{-x^2}dx\nonumber\\
	{}&=&  \frac{1}{\pi^{\frac{1}{2} } }\int_{-\infty}^{\infty}K\cos(x)e^{-x^2}dx\nonumber\\
	{}&=&  \frac{K}{\sqrt{\pi}e^{\frac{1}{4} } }\textrm{Re}\left(\int_{-\infty}^{\infty}e^{-(x-\frac{i}{2})^2}dx\right)\nonumber\\
	{}&=&  \frac{K}{e^{\frac{1}{4}}} = 0.77880\cdot K\label{eq:firstmom}
\end{eqnarray}

This result was mainly mentioned because it serves as a test to the software used; If one kicks a harmonic oscillator in the 
right way, then the shift in momentum should be exactly the above constant. Furthermore, it serves as a rough estimate on the orders of 
magnitude of the momentum shift a kick introduces. From this simple calculation, we can then deduce, that the fundamental effect a kick has 
on a Bose Einstein condensate is to shift its momentum slightly to the right of momentum space, by an amount that is proportional to the 
Kick strength. One can also see that a symmetric kick, e.g. $e^{-iK\cos x}$ would have had no effect on the mean momentum at all.

Furthermore, we want for the sake of completeness give an analytic estimate for $\langle p^2 \rangle$, since it
will be important when considering the mean width of the distribution, calculated by $\langle p^2 \rangle -
\langle p \rangle^2$. We thus proceed along the same lines:

\begin{eqnarray}
\langle p^2 \rangle &=&
-\frac{1}{\pi^{\frac{1}{2}}}\int_{-\infty}^{\infty}e^{-x^2}\left[(-x+iK\cos(x))^2+(-1-iK\sin(x))\right]\nonumber\\
		{}&=&\frac{K^2+e(K^2-1)}{2e}+1\label{eq:secmom}
\end{eqnarray}
After a somewhat lengthy calculation that proceeds as the one for~(\ref{eq:firstmom}), dropping the sine term and
integrating the rest as was shown before.
The variance can now be given as:
\begin{equation}\label{eq:variance}
	 \langle p^2 \rangle-\langle p \rangle^2=\frac{e+(1-2\sqrt{e}+e)K^2}{2e}=\sigma^2
\end{equation}
For a normal BEC in a trap having no interaction, this yields a variance of 0.577409 at kick strength $K=1$, which
was confirmed by a numerical computation of width and mean of the distribution in figure~\ref{fig:confirm}. A point on units has to be made. As I am using a fast Fourier Transform to calculate momentum distributions, the momenta vary from $-\frac{\pi N}{L}$ to $\frac{\pi N}{L}$, where $N$ is the number of points used and $L$ is the overall system size considered. I chose $N=1024$ and $L=10$ to be fitting values for our purposes. As mentioned, the spectrum of momenta is discrete and the unit of momentum would be $\frac{2\pi N}{LN}=\frac{\pi}{5}$. In order to get rid of factors of $\pi$, 
One can choose the momentum at the boudary of the Brillouin-zone, $q_B=\frac{40\pi}{L}$ as a unit of measurement. This yields a much more comfortable unit for momenta and where not stated otherwise, these units are used.

\begin{figure}[H]
\begin{center}
\includegraphics[scale=0.4]{figures/confirm}
\caption{A one dimensional Bose-Einstein Condensate kicked with a strength $K=1$. Note the slight shift to the
right. With the \texttt{analyze} Perl script, the mean momentum was found to be $\langle p \rangle=0.778801$, which in the above units is at rougly 0.06 and
$\langle p^2 \rangle-\langle p \rangle^2=0.577409$, which would be about 0.05. Although no Brillouin-zone is present, I chose to normalize
with the Brillouin vector, in order to get rid of factors of $\pi$.}
\label{fig:confirm}
\end{center}
\end{figure}

\begin{figure}[H]
\begin{center}
\includegraphics[scale=0.4]{figures/confirm2}
\caption{A one-dimensional BEC kicked with $K=5$, $\langle p \rangle\approx0.19$ and $\langle p^2\rangle - \langle p \rangle^2\approx 0.09$}
\label{fig:confirm2}
\end{center}
\end{figure}
