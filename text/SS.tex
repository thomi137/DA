\section{General Considerations}
We want to propagate the Gross-Pitaevskii equation in time. Therefore we need an efficient algorithm to do so. The method of choice was the so-called 
split-step operator method, which is based on the fact that a \emph{time-independent} Hamiltonian has an exact solution given by the time-evolution 
operator.
The idea is to take apart the Hamiltonian in two parts.
\begin{equation}\label{eq:ssidea}
	i\frac{\partial\psi(x, t)}{\partial t}=\left(-\frac{1}{2}\frac{\partial^2}{\partial x^2}+V(x)\right)\psi=\left(\hat{A}+\hat{B}\right)\psi(x, t)
\end{equation}
where $\hat{A}=-\frac{1}{2}\frac{\partial^2}{\partial x^2}$ and $\hat{B}=V(x)$. With that, we can write the formal solution
of~(\ref{eq:ssidea}) as
\begin{equation}\label{eq:ssstep}
	\psi(x, t+\Delta t)=e^{-i\Delta t \left(\hat{A}+\hat{B}\right)}\psi(x, t)
\end{equation}
It has to be noted however, that this solution is no longer exact as $V$ becomes itself
time-dependent~\cite{bandraukshen2}, since then, the Hamiltonians at different times do no longer commute, i. e. energy is
no longer conserved, and $\hat{B}$ would have to be expanded about $t$ to get a more accurate approximation.
Following~\cite{bandraukshen2}, we now consider the quantity
\begin{equation}\label{eq:S}
	S(\hat{A}, \hat{B}, \lambda)= e^{\lambda\left(\hat{A}+\hat{B}\right)}
\end{equation}
The idea of the split-step operator method is to expand the quatity $S$ to some order that governs the global quality of the
method. We can make an expansion about $\lambda$:
\begin{equation}\label{eq:Sexp}
	S(\hat{A},\hat{B},\lambda)=S_1(\hat{A}, \hat{B}, \lambda)+\frac{1}{2}\left[\hat{A},\hat{B}\right]\lambda^2+\ldots
\end{equation}
Where $S_1(\hat{A}, \hat{B}, \lambda)=e^{\lambda\hat{A}}e^{\lambda\hat{B}}$
One can further show~\cite{bandraukshen2, suzuki1} that:
\begin{equation}\label{eq:S2}
	S(\hat{A}, \hat{B},\lambda)=S_2(\hat{A}, \hat{B},
	\lambda)+\frac{1}{24}\left[\hat{A}+2\hat{B},\left[A+B\right]\right]\lambda^3+\mathcal{O}(\lambda^4)
\end{equation} 
with $S_2(\hat{A}, \hat{B}, \lambda)=e^{\lambda\hat{A}/2}e^{\lambda\hat{B}}e^{\lambda\hat{A}/2}$, which yields an approximation to
second order using a symmetric decomposition. Going to higher orders usually involves finding imaginary roots of polynomials, which
would introduce real quantities in the exponentials and consequently, those schemes do not conserve
unitarity~\cite{bandraukshen2}. This paper also shows, that the quantity $S_2$ is actually accurate to third order for
time-independent Hamiltonians, which stems from the fact that in this case the operators $\hat{A}$ and $\hat{B}$ do commute at
different times. 
I found it not to be practical to go beyond the third-order approximation, since the main effect of the higher accuracy is to slow down the calculations, and one can 
well account for higher accuracy by simply cutting down the time step. This practice was also adopted by recent work in the field~\cite{gardinerjaksch}. 

\section{The Gross-Pitaevskii Equation}
\subsection{The Framework}
We want to solve a nonlinear partial differential equation of the form:
\begin{equation}\label{eq:generalform}
	i\frac{\partial \psi}{\partial t}=(\hat{A}+\hat{B})\psi
\end{equation}
Where $\hat{A}$ contains all the spatial derivatives and $\hat{B}$ contains the potential terms that are in general functions of x and t. However, 
for time-independent potentials the exact solution of~(\ref{eq:generalform}) would be:
\begin{equation}\label{eq:exactsol}
	\psi(x, t+\Delta t)=e^{\lambda(\hat{A}+\hat{B})}\psi(x, t)
\end{equation}
Where $\lambda=i\Delta t$. Considering the above formulae, we are therefore confronted with the problem of splitting this
time-independent exponential.~\cite{bandraukshen3} states that for such a problem, a symmetric product formula as the above
to accuracy $\lambda^{2n+1}$ can be given by the recursion scheme
\begin{eqnarray}
	e^{\lambda(\hat{A}+\hat{B})}&=&S_{2n+1}+\mathcal{O}(\lambda^{2n+1})\label{eq:recscheme}\\
	S_{2n+1}(\lambda)&=&S_{2n-1}(\lambda s_{2n+1})S_{2n-1}\left[\lambda(1-2s_{n+1})\right]S_{2n-1}(\lambda s_{2n+1})\,
	,\nonumber\\
	2(s_{2n+1})^{2n-1}+(1-2s_{2n+1})^{2n-1}&=&0\, ,\label{eq:polynomial}\\
	S_3(\lambda)&=&e^{\lambda\frac{\hat{A}}{2}}e^{\lambda\hat{B}}e^{\lambda\frac{\hat{A}}{2}}\label{eq:s3}\nonumber
\end{eqnarray}
now adapting the index to the overall accuracy for clarity. It has to be mentioned that these approximants have the advantage of
being unitary, $S(\lambda)S(-\lambda)=1$ and therefore conserve the norm of the function $\psi(x, t)$ at every time step. The
roots of the polynomial~(\ref{eq:polynomial}) are such that certain commutators of $\hat{A}$ and $\hat{B}$ vanish. These roots can
be imaginary when considering schemes of arbitrary precision and thus can introduce some real exponential which does no longer
conserve the norm and is therefore not very practicable for an imaginary differential equation.
It is also important to note that $S_3$ already implies the calculation of three exponentials, $S_5$ has seven of them and $S_7$
has 19. Thus, the computational time needed diverges.~\cite{bandraukshen3} further shows that the accuracy for a given time step
$\lambda$ can be estimated to $\approx 10^{-6}$, for $S_3$.  
Our scheme used is therefore given by:
\begin{equation}\label{eq:ourcscheme}
	e^{\lambda(\hat{A}+\hat{B})}\approx S_3+\mathcal{O}(\lambda^3)
\end{equation}
Still, this leaves us with the problem of calculating the exponential of an operator containing derivatives. Happily however, the Fourier-transform diagonalizes 
the operator $e^{\frac{\partial^2}{\partial x^2}}$, and we can carry out the calculation in Fourier space rather than having to carry out a matrix multiplication, 
so the scheme ist of order $N\log N$ in the number of used points. We are now showing how to take a nonlinear Schr\"odinger equation apart.
We start with:
\begin{eqnarray}
	i\frac{\partial \psi}{\partial t}&=&-\frac{\partial^2\psi}{\partial x^2}+g|\psi|^2\psi+V(x)\psi\label{eq:gpeq}\\
	\psi(x, 0)&=&f(x)
\end{eqnarray}
By the special nature of the kick, we can regard the second term of the right-hand side of~(\ref{eq:gpeq}) as a time-independent potential, as only phase-factors are multiplied in, so a more accurate approximation is not needed.
For a discretized system, the solution to the above equation to third order in $\Delta t$ is given by:
\begin{equation}\label{eq:thirdordersol}
	\psi(x, t+\Delta t)=e^{-i\int_t^{t+\Delta t}\left(-\frac{\partial^2}{\partial x^2}+g|\psi|^2+V(x)\right)dt}\psi(x, t)+\mathcal{O}(\Delta t^3)
\end{equation}
Then, the exponential can be approximated to second order as:
\begin{equation}\label{eq:exapprox}
	e^{-i\int_t^{t+\Delta t}\left(-\frac{\partial^2}{\partial x^2}+g|\psi|^2+V(x)\right)dt}=e^{-i\Delta t \left(-\frac{\partial^2}{\partial x^2}+g|\psi|^2+V(x)\right)}
\end{equation}
This is now our workable solution to propagating the Gross-Pitaevskii equation and forms the very fundament of every calculation done.

\subsection{Finding the Groundstate Using the Split-Step Operator Method}
To calculate the Groundstate of a Bose-Einstein condensate in an optical lattice, one has several possibilities. The method used was
first to cast the Hamiltonian into its matrix representation, in order to diagonalize it using the LAPACK subroutine DSTEV. However,
this proved not favourable when considering large couplings, where the system would only slowly converge, if convergence occurred at all,
and each step involved diagonalizing a whole matrix of linear dimension $\sim 1024$, while only the lowest eigenvector was needed.
Mainly for this shortcoming, another technique was implemented. When using the split-operator technique as described above and propagating
in imaginary time, $\tau=it$, the system is obviously no longer normalized. However when renormalizing after each imaginary time step,
one gradually approaches the exact groundstate. Up to coupling strengths of 10, 20000 iterations proved sufficient to get a
groundstate that would by later propagation not alter its nature, nor its momentum profile. In this manner, one always applies
the same operator to the wavefunction, even when
propagating in real time, which of course adds to numerical stability.
 Recent work uses more sophisticated methods such as DDS~\cite{kostrun},
and Collocation with Legendre-Polynomials~\cite{choi}, mainly for the fact that the scheme presented here does not scale very well and is actually quite slow when going to higher
dimensions. 

\subsubsection{Propagation in Imaginary Time}
The method of choice applies predominantly to linear systems, yet it can be shown that one can use the same technique for nonlinear
Schr\"odinger equations (NLSEs). We start with the linear case. Our procedure is to apply the operator
$\mathcal{T}(\tau)=e^{-\tau\hat{H}}$ to some given initial state $|\psi_0\rangle$. The obtained state is subsequently normalized and we
optain an iterative procedure:
	\begin{equation}\label{eq:iter}
		|\psi_{n+1}\rangle =
		\frac{\mathcal{T}(\tau)|\psi_n\rangle}{\sqrt{\langle\psi_n\mathcal{T}(\tau)|\mathcal{T}(\tau)\psi_n\rangle}}
	\end{equation}
Let us now assume that the eigenstates to the Hamiltonian are known, so that we may express $|\psi_n\rangle$ in terms of these
eigenstates, $|\psi_n\rangle=\sum_ic_i|\phi_i\rangle$, and that the corresponding eigenvalues are ordered such that
$|\lambda_0|<|\lambda_1|<\ldots$. Then, the right-hand side of~(\ref{eq:iter}) can be cast in the form:
	\begin{eqnarray}
		\frac{\mathcal{T}(\tau)|\psi_n\rangle}{\sqrt{\langle\psi_n\mathcal{T}(\tau)|\mathcal{T}(\tau)\psi_n\rangle}}
		&=&\frac{\sum_ic_i\mathcal{T}(\tau)|\phi_i\rangle}{\sqrt{\sum_{i,
		j}c_i^*c_j\langle\phi_i|\mathcal{T}(\tau)\mathcal{T}(\tau)|\phi_j\rangle}}\nonumber\\
		{}&=&\frac{\sum_ic_ie^{-\tau\lambda_i}|\phi_i\rangle}{\sqrt{\sum_{i}|c_i|^2e^{-2\tau\lambda_i}}}\nonumber\\		
		{}&=&\frac{\sum_ic_ie^{-\tau\lambda_i}|\phi_i\rangle}{\sqrt{\sum_{i}|c_i|^2e^{-2\tau\lambda_i}}}\cdot\frac{e^{\tau\lambda_0}}{e^{\tau\lambda_0}}\nonumber\\
		{}&=&\frac{\sum_ic_ie^{-\tau(\lambda_i-\lambda_0)}|\phi_i\rangle}{\sqrt{\sum_{i}|c_i|^2e^{-2\tau(\lambda_i-\lambda_0)}}}\nonumber
	\end{eqnarray}
Thus, since by our assumption the exponents remain negative for all $i\neq 0$, one can readily see that this procedure amplifies
the groundstate $\phi_0$, so that, as $\tau\rightarrow\infty$, the system will be in its groundstate. This proof does however
only apply to the linear case. Its nonlinear generalization requires an additional component to the ground state, 
$|\tilde{\phi}\rangle$, obviously satisfying $\langle\tilde{\phi}|\hat{H}|\tilde{\phi}\rangle\leq\lambda_1$, since then, our method is still valid, with the assumption that there is a gap between this state and the first excited state, and therefore a discrete spectrum even for the nonlinear case. It can readily be shown, that the energy levels of the Gross-Pitaevskii equation are indeed discrete\footnote{For a proof for sufficiently smooth potentials, the reader is referred to~\cite{konotop}}, so our method is well applicable here. Using this method, one
effectively finds the groundstate for arbitrary coupling constants.
	
	
\section{The Code}
\subsection{C++-Code for the Split-Step Operator Method}
One of the main goals of this work was to obtain working code that is sufficiently fast to yield results with a decent accuracy.  All the papers cited so far exhibit their results, yet few do comment on the
method or code used, which slows down the overall implementation since one has to reinvent the wheel. Most of the code accompanying this work was used for testing,
particularly the code in the oldcode directory, which was used to develop a feeling for the split-step operator method. The main Files needed are
\texttt{split\_step.h}, 
\texttt{BEC\_Groundstate.h} and \texttt{fft.h}. The first two implement templated functions and classes to do the timesteps and to find the Groundstate. Provision is made
for the Groundstate solver to use the Lanczos-Algorithm of the \href{http://www.comp-phys.org/software/ietl/ietl.html}{IETL}, which can be introduced by simply
implementing the functions needed. The file \texttt{fft.h} contains a C++ wrapper function for the well known \href{http://www.fftw.org}{FFTW}-library to execute
fourier Transforms. The code will not compile on a system, where fftw Version 3.0.1 is not installed. Also, the code makes use of some data-structures from the
\href{http://www.boost.org}{Boost}-library, namely uBlas-matrices for the LAPACK routines. At present however, the programs can do without, when finding the
groundstate with the relaxation method.
The working of the programs is in short summarized by the following steps:
\begin{enumerate}
	\item Parse command line arguments (kick strength, coupling strength, maximum time, $\Delta t$, number of points) if given.
	\item Find the groundstate and normalize.
	\item Kick as needed.
	\item 
		\textbf{for} t$<$maxt, ++t \textbf{do}:\\
			\,split\_step\\
			\,\textbf{if} \textbf{not}(t\%period)\\
				\,\,print data\\
			\,\textbf{fi}\\
		\textbf{done}\\
\end{enumerate}
The output is directed to standard output, whence it can be redirected to any file by using the shell of choice.
\subsection{Perl code}
In order to optimize the calculation time used on the \href{http://www.asgard.ethz.ch}{Asgard} Beowulf cluster of the ETH, I wrote code that printed various data. In
order to analyze these files, I wrote a small perl script, \texttt{analyze}, which allowed to do some statistics with calculated histograms. The script
\texttt{scripter.pl} was used to create PBS jobs on Asgard, taking as command-line arguments the processors needed, the number of nodes and the walltime. The analysis
of the histogram data is treated further on in the text. 


