\section{The Lattice}
The consequences of a BEC being enclosed in a lattice were mentioned before. Experimentally however, the kick on the condensate is exerted by tilting the trap out of the axis for a short time. Therefore, apart from the lattice, we also have to add a trap. This however is just the superposition of 
two potential terms. 
The lattice in the potential was added by simply adding the term $\frac{q_B^2}{2}\sin^2(q_Bx)$\cite{menotti}, where $q_B$ is the wavevector 
at the boundary of the first Brillouin zone, $q_B=\frac{\pi}{a}$ and $a$ is the lattice constent. The Groundstate of such a potential is shown 
in the figures below. 
The form of the lattice is not arbitrary. The factor $\frac{q_B^2}{2}$ arises, because of the recoil energy an atom acquires when absorbing a 
photon with momentum $q$. As the field produced by a laser which is normally used to produce a lattice is $E_0\sin(qz)\sin(\omega t)$, the time-averaged 
field is then proportional to $\sin^2(qz)$\cite{pitaevskii}.
It has to be noted, that the dynamics are non-trivial, because of the nonlinearity that arises from the interatomic interactions. It is most fitting to 
use numerical techniques.\cite{menotti} shows that for deep enough potentials, the system undergoes a transition from the superfluid regime to a Mott-insulator 
state, as can be seen by considering a tight binding approximation. The following pictures will illustrate the nature of the Ground state further. 
Let now the potential be of the form above, and let the atoms be trapped with the normal harmonic potential. The overall external potential is then periodic. 
Let again $\psi(x)$ be the wavefunction of a one dimensional Bose-Einstein condensate still obeying the equation~(\ref{eq:GrossPitaevskii}). As was explained in
chapter~\ref{ch:BEC}
Figures~\ref{fig:WFk0g1} and~\ref{fig:WFk0g5} show the first two sattelite peaks, that 
are obtained in precisely this way
\begin{figure}[H]
\begin{center}
\includegraphics[scale=0.4]{figures/WFk0g1}
\caption{Ground state distribution of a Bose-Einstein condensate in a periodic potential and its corresponding momentum distribution with
		$g=1$}
\label{fig:WFk0g1}
\end{center}
\end{figure}


\begin{figure}[H]
\begin{center}
\includegraphics[scale=0.4]{figures/WFk0g5}
\caption{Same as figure \ref{fig:WFk0g1} with $g=5$}
\label{fig:WFk0g5}
\end{center}
\end{figure}

